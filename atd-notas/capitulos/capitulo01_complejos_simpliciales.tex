\documentclass{standalone}


\begin{document}
	En este capítulo seguimos \cite{alberto:2023} y \cite{kozlov:2008:combinatorial:alg:topo} como introducción a los complejos simpliciales.
	\noindent Para cualquier conjunto $V$, su conjunto potencia es denotado por $\mathcal{P}(V)$. La carnalidad de $V$ es $\# V$
	
	\begin{definition}\label{def:simplicial_complex}
		Un \emph{complejo simplicial} sobre un conjunto $V$ es un conjunto finito $K\subseteq\mathcal{P}(V)$ cerrado bajo subconjuntos.
	\end{definition}
	
	Formalmente, la definición de arriba corresponde a la de los \emph{complejos simpliciales abstractos y finitos} \cite[Definition 2.1]{kozlov:2008:combinatorial:alg:topo}; dado que no estudiaremos otro tipo de complejos simpliciales omitimos los otros adjetivos. 
	
	\begin{remark}\label{rem:empty_complexes}
		En la Definición~\ref{def:simplicial_complex} tenemos dos complejos simpliciales que son importantes. El primero es el complejo simplicial \emph{vano} (\emph{void} en inglés), es decir $\emptyset\subseteq\mathcal{P}(V)$. El segundo es el complejo simplicial \emph{vacío} (\emph{empty} en inglés): $\{\emptyset\}$ \cite[Remark 2.3]{kozlov:2008:combinatorial:alg:topo}. 
	\end{remark}
	
	\section{Notación y nomenclatura estándar}
	
	\noindent Sea $K$ un complejo simplicial. Cada elemento de $K$ se llama \emph{simplejo}\footnote{También es común en español, quizás lo es más, utilizar el término \emph{cara}. No lo adopto porque viene del uso geométrico de esa palabra: \textit{las caras del poliedro.} Aunque es verdad que los complejos simpliciales pueden considerarse objetos geométricos, su combinatoria es más natural. Véanse \cite{kozlov:2008:combinatorial:alg:topo}, \cite{may:1967:simplicial}} (\emph{simplex} y plural \emph{simplices} en inglés). La \emph{dimensión} de $\sigma$ es $\dim(\sigma)=\#\sigma-1$. Si $\dim(\sigma)=k$ decimos que $\sigma$ es un \emph{$k$-simplejo}. Un \emph{vértice} de un complejo simplicial es un $0$-simplejo; El conjunto de  $n$-simplejos de $K$ será denotado por $K_{n}$. 
	
	La dimensión de $K$ es $\dim(K)=\max\{\dim(\sigma)\mid\sigma\in K\}$. Una \emph{faceta} (o \emph{careta} si se usa cara) de un complejo simplicial es un simplejo máximal con respecto a la contención; es decir, $\sigma$ es faceta si $\sigma\subseteq\tau$, entonces $\tau=\sigma$. Diremos que un complejo simplicial es \emph{puro} wsiempre que todas sus facetas tengan la misma dimensión.
	
	Los complejos simpliciales son objectos combinatorios. Usualmente se representan gráficamente como sigue. Un punto representa un vértice, las aristas,  $1$-simplejos; los triángulos representan $2$-simplejos, etc. En la Figura~\ref{fig:eg_simplex} encontramos ejemplos de simplejos y en la Figura~\ref{fig:eg_simplicial_complex}, de complejos simpliciales. Nótese que el triángulo hueco en la Figura~\ref{fig:eg_non_pure} no es un simplejo. También, es importante notar que del dibujo solo es imposible determinar si el tetraedro de la Figura~\ref{fig:3_simplex} es un $3$-simplejo o es la unión de 4 $2$-simplejos. Es por esto último que sólo representamos con fines ilustrativos.
	
	\begin{figure}[h]
		\centering
		\begin{subfigure}{.3\textwidth}
			\centering
			\subimport{images/}{eg_1_simplex}
			\caption{Un $1$-simplejo.}
			\label{fig:1_simplex}
		\end{subfigure}
		\begin{subfigure}{.3\textwidth}
			\centering
			\subimport{images/}{eg_2_simplex}
			\caption{Un $2$-simplejo.}
			\label{fig:2_simplex}
		\end{subfigure}
		\begin{subfigure}{.3\textwidth}
			\centering
			\subimport{images/}{eg_3_simplex}
			\caption{Un $3$-simplejo.}
			\label{fig:3_simplex}
		\end{subfigure}
		\caption{Representación gráfica de alguno simplejos. el tetraedro en \subref{fig:3_simplex} es un sólido $3$-dimensional.}
		\label{fig:eg_simplex}
	\end{figure}
	\begin{figure}[h]
		\centering
		\begin{subfigure}{.4\textwidth}
			\centering
			\subimport{images/}{eg_octahedral_sphere}
			\caption{A pure simplicial complex.}
			\label{fig:eg_octahedral_sphere}
		\end{subfigure}
		\begin{subfigure}{.4\textwidth}
			\centering
			\subimport{images/}{eg_non_pure_simplicial_complex}
			\caption{A non-pure simplicial complex.}
			\label{fig:eg_non_pure}
		\end{subfigure}
		\caption{Examples of simplicial complexes. All triangles in \subref{fig:eg_octahedral_sphere} are filled but the simplicial complex itself is not a $3$-dimensional solid.}
		\label{fig:eg_simplicial_complex}
	\end{figure}
	
	Sea $\sigma$ un simplejo. Vamos a sobrecargar la notación y usaremos $\sigma$ para denotar al complejo simplicial $\mathcal{P}(\sigma)$ como a la única faceta de tal complejo simplicial. En este sentido, todo simplejo es un complejo simplicial.
	
	\begin{exercise}
		¿Cuántos $k$-simplejos tiene un $n$-simplejo?
	\end{exercise}
	
	El \emph{complejo simplicial generado} por $F\subseteq\mathcal{P}(V)$ es el complejo simplicial mínimo con respecto a contener todos los simplejos $\sigma\in F$. Tal complejo simplicial es denotado por 
	\[
	\bigcup F=\bigcup_{\sigma\in F}\sigma.
	\]
	
	\begin{exercise}
		Muestra que todo complejo simplicial es generado por sus facetas.
	\end{exercise}
	
	Un \emph{subcomplejo} de un complejo simplicial $K$ es un subconjunto de $K$ que es un complejo simplicial. El \emph{$n$-esqueleto} de $K$ es el subcomplejo de $K$ generado por todos los $k$-simplejos en $K$ con $k\leq n$ y es denotado por $\operatorname{skel}_{n}(K)$.
	
	\begin{exercise}
		Muestra que en general $\operatorname{skel}_{n}(K)$ no es puro de dimensión $n$; es decir, $\operatorname{skel}_{n}(K)$ puede contener facetas de dimensión $k < n$.
	\end{exercise}
	
	\begin{exercise}
		Muestra que $\operatorname{skel}_{n}(K)=K_{n}$ si y sólo sí $n=-1$ o $K$ es vano. ¿Qué pasaría con esta equivalencia si en la Definición~\ref{def:simplicial_complex} eliminamos los complejos simpliciales vacío y vano?
	\end{exercise}	
	
	Un \emph{simplejo frontera} de un $k$-simplejo $\sigma$, es un $(k-1)$-simplejo $\tau\in\sigma$. La \emph{frontera} de $\sigma$ es $\partial\sigma=\bigcup\{\tau\mid\tau\text{ es un simplejo frontera de }\sigma\}$.
	
	\begin{exercise}
		$\partial\sigma$ es un complejo simplicial.
	\end{exercise}
	
	\section{Realización geométrica}
	\noindent Intuitivamente, el complejo simplicial de la Figura~\ref{fig:eg_octahedral_sphere} es una $2$-esfera. Esta sección formaliza tal intuición. Usamos $\mathbb{R}^{V}$ para denotar el $\mathbb{R}$ espacio vectorial con base $V$, en particular, $\mathbb{R}^{n}$ es el espacio euclidiano de $n$ dimensiones. Siempre consideramos $\mathbb{R}^{V}$ como espacio topológico con la topología inducida por la métrica euclidiana.
	
	\begin{definition}\label{def:affine_ind_set}
		Un \emph{conjunto afinmente independiente} en $\mathbb{R}^{n}$ es un conjunto $A=\{v_{0},\ldots,v_{k}\}$ tal que para cada $t_{i},s_{i}\geq0$ con 
		\[
		\sum_{i=0}^{k}s_{i}=\sum_{i=0}^{k}t_{i}=1,
		\]si 
		\[
		\sum_{i=0}^{k}s_{i}v_{i}=\sum_{i=0}^{k}t_{i}v_{i},
		\]entonces $s_{i}=t_{i}$ para cada $i$. Un vector $x\in\mathbb{R}^{n}$ es una \emph{combinación afín} de $A$ si $x=\sum_{i=0}^{k}s_{i}v_{i}$ con $\sum_{i=0}^{k}s_{i}=1$ y $s_{i}\geq0$ para cada $i$. La \emph{cápsula convexa} (\emph{convex hull} en inglés) de $A$ es el conjunto de todas las combinaciones afines de $A$.
	\end{definition}
	
	\begin{exercise}
		Demuestra que todo conjunto linealmente independiente de $\mathbb{R}^{n}$ es afín independiente.
	\end{exercise}
	
	Tomamos las definiciones~\ref{def:standard_V_simplex} y~\ref{def:geom_real} de \cite[Section 2.2.1]{kozlov:2008:combinatorial:alg:topo}.
	
	\begin{definition}\label{def:standard_V_simplex}
		Sea $V$ un conjunto finito. El \emph{$V$-simplejo estándar} es la cápsula convexa de la base ortnonormal estándar de $\mathbb{R}^{V}$. 
	\end{definition}
	En la Figura~\ref{fig:eg_standard_2_simplex}, tenemos el $V$-simplejo estándar para $V=\{x_{0},x_{1},x_{2}\}$. 
	\begin{figure}
		\centering
		\import{images/}{eg_standard_2_simplex}
		\caption{El $V$-simplejo estándar para $V=\{x_{0,}x_{1},x_{2}\}$.}
		\label{fig:eg_standard_2_simplex}
	\end{figure}
	
	\begin{definition}\label{def:geom_real}
		Sea $K$ un complejo simplicial. Consideremos la unión de todos los $\sigma$-simplejos estándar en $\mathbb{R}^{K_{0}}$ con $\sigma\in K$. Este conjunto con la topología de subespacio de $\mathbb{R}^{K_{0}}$ es la \emph{realización geométrica estándar} de $K$.  Cualquier espacio homeomorfo a la realización geométrica estándar de $K$ como $|K|$ y la llamaremos la \emph{realización geométrica} de $K$.
	\end{definition}
	
	\begin{remark}\label{rem:geom_realization}
		La realización geométrica es definida mediante los $V$-simplejos estándar. Como ellos tienen la topología de subespacio de $\mathbb{R}^{V}$ siempre supondremos que $|K|$ tiene una métrica que le induce la topología adecuada. También, cualquier afirmación topológica hecha sobre un complejo simplicial estará hecha sobre su realización geométrica.. 
	\end{remark}
	
	\begin{exercise}\label{rem:geom_realization_colimit}
		Sea $K$ un complejo simplicial. En este ejercicio veremos cómo podemos estudiar complejos simpliciales infinitos y sus realizaciones geométricas. La equivalencia de la última parte sirve como definición de la topología de la realización geométrica de complejos simpliciales infinitos \cite[p. 8]{munkres:1984:algebraic:topology}.
		\begin{enumerate}
			\item Prueba que si $\sigma\in K$, entonces $|\sigma|\subseteq |K|$.
			\item Muestra que la realización geométrica de cualquier simplejo es un espacio compacto.
			\item Demuestra que si $\sigma\in K$, entonces $|\sigma|$ es un conjunto cerrado de $|K|$.
			\item Demuestra que $C\subseteq|K|$ es cerrado si y solo si $C\cap|\sigma|$ es cerrado para cada $\sigma\in K$.
		\end{enumerate}
	\end{exercise}
	
	\section{Funciones simpliciales}
	\noindent Si $f\colon X\rightarrow Y$ es una función, usamos $f(A)$ para denotar la imagen de $A\subseteq X$ bajo $f$.
	\begin{definition}\label{def:simplicial_map}
		Dados dos complejos simpliciales $K$ and $L$, una \emph{función simplicial} \footnote{Es usual en la literatura en inglés encontrar esto como \emph{simplicial map}. Yo mismo les llamo así a veces.}de $K$ a $L$ es una función $f\colon K_{0}\rightarrow L_{0}$ tal que si $\sigma\in K$ entonces $f(\sigma)\in L$.
	\end{definition}
	\begin{remark}
		Toda función simplicial $f\colon K_{0}\rightarrow L_{0}$ induce una función $f\colon K\rightarrow L$. Así, cuando decimos que $f\colon K\rightarrow L$ es una función simplicial, lo que queremos decir es que es inducida por un mapa simplicial. 
	\end{remark}
	
	\begin{exercise}
		Demuestre que la clase de complejos simpliciales $\mathcal{CS}$ junto con las funciones simpliciales es una categoría; es decir, demuestre que la composición de funciones simpliciales es simplicial y que para cada $K\in\mathcal{CS}$ existe una función simplicial $1_{K} \colon K\rightarrow K$ tal que $1_{K}\circ f = f$ y $g\circ 1_{K} = g$ para cualesquiera funciones simpliciales $f\colon L\rightarrow K$ y $g\colon K\rightarrow L$.
	\end{exercise}
	
	\begin{exercise}
		¿Quiénes son $1_{\emptyset}$ y $1_{\{\emptyset\}}$? ¿Tu prueba del ejercicio anterior consideró esto?
	\end{exercise}
	
	\begin{exercise}\label{rem:coeherent_gluing}
		Muestra que no toda función entre los vértices de dos complejos simpliciales induce una función simplicial. 
	\end{exercise}
	
	\begin{exercise}
		Muestra que toda función simplicial $f\colon K\rightarrow L$ induce una función continua $|f|\colon |K|\rightarrow |L|$ dada por 
		\[
		|f|(\sum_{i=0}^{k}s_{i}v_{i})= \sum_{i=0}^{k}s_{i}f(v_{i}).
		\]
	\end{exercise}
	
	\section{Subdivisiones}
	\begin{definition}\label{def:subdivision}
		Una \emph{subdivisión} de un complejo simplicial $K$ es un complejo simplicial $K'$ tal que existen subdivisiones $|K|$ y $|K'|$ tales que:
		
		\begin{enumerate}
			\item La realización geométrica de cada simplejo en $K'$ está contenida en la realización geométrica de algún simplejo de $K$.
			\item Para cada $\sigma\in K$, hay un subcomplejo $\sigma'$ de $K'$ tal que $|\sigma|=|\sigma'|$.
		\end{enumerate}
	\end{definition}
	
	Los siguientes ejercicios están diseñados para una mejor y pronta comprensión de la definición de arriba.
	
	\begin{exercise}
		Si $K'$ es una subdivisión de $K$, entonces $|K|\cong|K'|$.
	\end{exercise}
	
	\begin{exercise}
		Representa gráficamente los siguientes complejos simpliciales
		\begin{enumerate}
			\item  $K=\{\emptyset, \{a\}, \{b\}, \{a,b\}\}\}$.
			\item  $K'=\{\emptyset, \{a\}, \{b\}, \{c\}, \{a,c\}, \{c,b\}\}\}$
		\end{enumerate}
		Muestra que $K'$ es una subdivisión de $K$.
	\end{exercise}
	
	\begin{exercise}
		Muestra que la imagen en la Figura~\ref{fig:eg_barycentric} es una subdivisión de un $2$-simplejo.
	\end{exercise}
	
	\begin{figure}
		\centering
		\import{images/}{eg_barycentric}
		\caption{La primera subdivisión baricéntrica de un $2$-simplejo.}
		\label{fig:eg_barycentric}
	\end{figure}
	
	\begin{figure}
		\centering
		\import{images/}{eg_non_subdivision}
		\caption{Esto no es una subdivisión de un $2$-simplejo.}
		\label{fig:eg_non_subdivision}
	\end{figure}
	
	Intuitivamente, una subdivisión de $K$ se obtiene agregando simplejos internos a (algunos de) los simplejos de $K$. Sin embargo, esto debe hacerse con cuidado. En la Figura~\ref{fig:eg_non_subdivision} vemos que este procedimiento fue realizado sin \textbf{preservar la estructura combinatoria} del complejo simplicial original. Esto quiere decir que aunque el espacio topológico asociado al objeto en esa figura (es decir el triángulo) es la realización geométrica del $2$-simplejo, como objeto combinatorio no es un complejo simplicial. 
	
	\begin{exercise}
		Demuestra que si $L$ es una subdivisión de $K$ y $K$ lo es de $L$, entonces existe un isomorfismo $K\cong L$.
	\end{exercise}

	Aunque existen diferentes subdivisiones generadas con un método, nosotros solamente estudiaremos la subdivisión baricéntrica. Para ello necesitamos algunas definiciones relacionadas a órdenes parciales.
	
	\begin{definition}\label{defn:poset}
		Un \emph{conjunto parcialmente ordenado} (abreviado \emph{orden parcial}) es una pareja $(P,<)$ donde $P$ es un conjunto y $<$ es una relación binaria sobre $P$ que es transitiva, irreflexiva y asimétrica. Simplificaremos esto y diremos que $P$ es un orden parcial y $<$ es el \emph{orden} de $P$. Una \emph{cadena} $\sigma$ en un orden parcial $P$ es subconjunto de $P$ tal que la restricción de $<$ a $
		\sigma$ satisface \emph{tricotomía}, es decir $(C,<)$ es un \emph{conjunto totalmente ordenado} (abreviado \emph{orden total}).
	\end{definition}
	Aunque no profundizaremos en la teoría de órdenes parciales, para poder entender más fácilmente qué es el complejo de orden de un orden parcial, usaremos la representación gráfica de los órdenes parciales. 
	\begin{definition}\label{defn:Hasse}
		Sean $P$ un orden parcial y $x<y$ en $P$. Decimos que $y$ \emph{cubre} $x$ si no existe $z\in P$ tal que $x<z<y$. El \emph{diagrama de Hasse} $\mathcal{H}$ de $P$ es la gráfica con conjunto de vértices $P$ y aristas definidos por pares ${x,y}$ tales que $x$ cubre $y$ o viceversa.
	\end{definition}
	
	\begin{figure}[h]
		\centering
		\begin{subfigure}{.3\textwidth}
			\centering
			\import{images/}{eg_hasse_graded_poset}
			\caption{$a<b$, $c<d$}
			\label{fig:hasse_graded_poset}
		\end{subfigure}
		\begin{subfigure}{.3\textwidth}
			\centering
			\import{images/}{eg_hasse_graded_poset_min}
			\caption{$x<a$, $x<c$}
			\label{fig:hasse_graded_poset_min}
		\end{subfigure}
		\begin{subfigure}{.3\textwidth}
			\centering
			\import{images/}{eg_hasse_graded_poset_shell}
			\caption{$c<b$}
			\label{fig:hasse_graded_poset_shell}
		\end{subfigure}
		\caption{Diagramas de Hasse de tres órdenes parciales. En \subref{fig:hasse_graded_poset} tenemos dos relaciones de cobertura solamente. Agregamos dos relaciones más y obtenemos \subref{fig:hasse_graded_poset_min}. Con una relación más se obtiene \subref{fig:hasse_graded_poset_shell}}
		\label{fig:hasse}
	\end{figure}
	
	
	\begin{definition}
		El \emph{complejo de orden} de un orden parcial $P$ es el complejo simplicial $\Delta(P)$ generado por las cadenas maximales de $P$. En otras palabras $\sigma\subseteq P$ es un simplejo de $\Delta(P)$ si y solo si es un orden total con el orden heredado de $P$.
	\end{definition}
	
	\begin{exercise}
		Dibuja los complejos de orden de los órdenes de la Figura~\ref{fig:hasse}.
	\end{exercise}
	
	\begin{definition}\label{def:barycentric_subdivision}
		La \emph{primera subdivisión baricéntrica} de un complejo simplicial $K$ es el complejo de orden de $K$ visto como orden parcial y lo denotamos mediante $\operatorname{Bar}(K)$. La \emph{$N$-ésima subdivisión baricéntrica} de $K$ es la primera subdivisión baricéntrica de $\operatorname{Bar}^{N-1}(K)$.
	\end{definition} 
	
	En otras palabras, los simplejos de $\operatorname{Bar}(K)$ son cadenas de simplejos en $K$. En la Figura~\ref{fig:eg_barycentric} encontramos la forma usual de representar gráficamente una subdivisión baricéntrica. El vértice central, el baricentro del triángulo, corresponde a la única faceta del $2$-simplejo. En general, el baricentro de un simplejo corresponde a su única faceta. 
	
	\section{Teorema de aproximación simplicial}
	
	El teorema de aproximación simplicial nos asegura que dada cualquier función continua $f$ entre las realizaciones geométricas de dos complejos simpliciales,  podemos encontrar una función simplicial de una subdivisión del dominio de $f$ al codominio de $f$ que se ``parece'' a $f$. Por supuesto, para formalizar esto necesitamos varias definiciones.
	
	\begin{definition}\label{def:diameter}
		El \emph{diámetro} $\operatorname{diam}(X)$ de un conjunto $X\subseteq\mathbb{R}^{k}$ es $\sup\{d(x,y)\mid x,y\in X\}$.
	\end{definition}
	
	Aunque no hemos definido qué es una operación para obtener subdivisiones, intuitivamente, son métodos que permiten subdividir cualquier simplejo iterativamente. Un ejemplo son las subdivisiones baricéntricas.
	
	\begin{definition}\label{def:mesh}
		Una operación para obtener subdivisiones $\operatorname{Div}$ \emph{refina} si para cada $\epsilon>0$ y cada complejo simplicial $K$ existe $N\in\mathbb{N}$ tal que \[\sup\{\operatorname{diam}(\sigma)\mid\sigma\in\operatorname{Div}^{N}(K)\}<\varepsilon.\]
	\end{definition}
	La prueba del teorema de aproximación simplicial necesita el número de Lebesgue de una cubierta de espacio compacto.
	
	\begin{definition}\label{def:lebesgue}
		Sean $C$ un subespacio compacto de $\mathbb{R}^{k}$ y $\mathcal{U}$ una cubierta abierta de $C$. Decimos que $\lambda>0$ es un \emph{número de Lebesgue} de $\mathcal{U}$ si para cada $x\in C$ y $\varepsilon<\lambda$ la bola abierta $B_{\epsilon}(x)$ está contenida en algún elemento de $\mathcal{U}$.
	\end{definition}
	
	\begin{definition}
		La \emph{estrella abierta} de un vértice $v$ de un complejo simplicial $K$ es $\operatorname{St}(v)=\bigcup_{v\in\sigma}|\sigma|^{\circ}$.
	\end{definition}
	
	\begin{exercise}\label{rem:stars}
		 Demuestra que $\bigcap_{v\in \sigma}\operatorname{St}(v)\neq\emptyset$ si y solo si $\sigma\in K$ 
	\end{exercise}
	Recordemos que nuestros complejos simpliciales son finitos. Una versión general del siguiente teorema puede encontrarse en \cite[Theorem 16.5]{munkres:1984:algebraic:topology} 
	\begin{theorem}[Aproximación simplicial]\label{thm:simp_app}
		Sea $f\colon|K|\rightarrow|L|$ una función continua entre las realizaciones geométricas de dos complejos simpliciales. Si $\operatorname{Div}$ refina, entonces existen $N\in\mathbb{N}$ y una función simplicial $\overline{f}\colon\operatorname{Div}^{N}(K)\rightarrow L$ tales que $f(\operatorname{St}(v))\subseteq\operatorname{St}(\overline{f}(v))$ para cada $v\in S_{0}(\operatorname{Div}^{N}(K))$.
	\end{theorem}
	
	\begin{proof}
		Supongamos que tenemos métricas para $|K|$ y $|L|$ que nos dan la topología adecuada. Las estrellas abiertas $\operatorname{St}(w)$ con $w\in L_{0}$ forman una cubierta abierta de $|L|$. Sea $\lambda$ un número de Lebesgue de esa cubierta. Dado que $f$ es continua y $|K|$ es compacto existe $\delta>0$ such that $f(B_{\delta}(x))\subseteq B_{\lambda}(f(x))\subseteq \operatorname{St}(w)$ para algún $w\in L_{0}$. Elijamos $N$ tal que $\sup\{\operatorname{diam}(\sigma)\mid\sigma\in\operatorname{Div}^{N}(K)\}<\frac{\delta}{2}$. Definamos $\overline{f}\colon\operatorname{Div}^{N}(K)\rightarrow L$ mediante $\overline{f}(v)=w$ para algún $w$ tal que $f(\operatorname{St}(v))\subseteq \operatorname{St}(w)$. Este es una función simplicial bien definida por Ejercicio~\ref{rem:stars}.
	\end{proof}
	Finalmente, tenemos que la subdivisión baricéntrica refina \cite[Theorem 15.4]{munkres:1984:algebraic:topology}; por tanto, podemos usar el teorema anterior con subdivisiones baricéntricas.
	\begin{proposition}\label{prop:bar_mesh}
		La subdivisión baricéntrica refina.
	\end{proposition}
	
	\begin{proof}
		Sea $\sigma$ un $k$-simplejo. Por definición $(\operatorname{Bar}(\sigma))_{0}=\{\tau\mid\tau\in \sigma\}$. Identificamos $\tau$ con $\sum_{v\in\tau}\frac{v}{\#\tau}\in|\sigma|$, en otras palabras $\tau$ es el baricentro de $|\tau|$.
		
		Para cada $v\in \sigma$, \begin{align}
			d(\sigma,v)=&d(\sum_{w\in\sigma}\frac{1}{k+1}w,\sum_{w\in\sigma'}\frac{1}{k+1}v)\\
			\leq& \sum_{w\in\sigma}d(\frac{1}{k+1}w,\frac{1}{k+1}v)\\
			=&\frac{1}{k+1}\sum_{w\in\sigma}d(w,v)\\
			\leq&\frac{k}{k+1}\sup\{d(w,v)\mid w\in\sigma\}\\
			=&\frac{k}{k+1}\operatorname{diam}(\sigma).		
		\end{align}
		Dado que $\operatorname{diam}(\sigma)$ se alcanca para algún par $v,w\in\sigma$, de los cálculos de arriba concluimos que $\sup\{\operatorname{diam}(\tau)\mid\tau\in\operatorname{Bar}(\sigma)\}\leq\frac{k}{k+1}\operatorname{diam}(\sigma)$. Debido a que $\frac{k}{k+1}<1$ la proposición queda probada.
	\end{proof}
	\begin{exercise}
		Muestra con todo detalle la Proposición~\ref{prop:bar_mesh}.
	\end{exercise}
\end{document}