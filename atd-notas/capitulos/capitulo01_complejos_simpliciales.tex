\documentclass{standalone}


\begin{document}
	En este capítulo seguimos \cite{alberto:2023} y \cite{kozlov:2008:combinatorial:alg:topo} como introducción a los complejos simpliciales.
	\noindent Para cualquier conjunto $V$, su conjunto potencia es denotado por $\mathcal{P}(V)$. La carnalidad de $V$ es $\# V$
	
	\begin{definition}\label{def:simplicial_complex}
		Un \emph{complejo simplicial} sobre un conjunto $V$ es un conjunto finito $K\subseteq\mathcal{P}(V)\setminus\{\emptyset\}$ cerrado bajo subconjuntos.
	\end{definition}
	
	Formalmente, la definición de arriba corresponde a la de los \emph{complejos simpliciales abstractos y finitos} \cite[Definition 2.1]{kozlov:2008:combinatorial:alg:topo}; dado que no estudiaremos otro tipo de complejos simpliciales omitimos los otros adjetivos. 
	
	\begin{remark}\label{rem:empty_complexes}
		En la Definición~\ref{def:simplicial_complex} tenemos dos complejos simpliciales que son importantes. El primero es el complejo simplicial \emph{vano} (\emph{void} en inglés), es decir $\emptyset\subseteq\mathcal{P}(V)$. El segundo es el complejo simplicial \emph{vacío} (\emph{empty} en inglés): $\{\emptyset\}$ \cite[Remark 2.3]{kozlov:2008:combinatorial:alg:topo}. 
	\end{remark}
	
	\section{Notación y nomenclatura estándar}
	
	\noindent Sea $K$ un complejo simplicial. Cada elemento de $K$ se llama \emph{simplejo}\footnote{También es común en español, quizás lo es más, utilizar el término \emph{cara}. No lo adopto porque viene del uso geométrico de esa palabra: \textit{las caras del poliedro.} Aunque es verdad que los complejos simpliciales pueden considerarse objetos geométricos, su combinatoria es más natural. Véanse \cite{kozlov:2008:combinatorial:alg:topo}, \cite{may:1967:simplicial}} (\emph{simplex} y plural \emph{simplices} en inglés). La \emph{dimensión} de $\sigma$ es $\dim(\sigma)=\#\sigma-1$. Si $\dim(\sigma)=k$ decimos que $\sigma$ es un \emph{$k$-simplejo}. Un \emph{vértice} de un complejo simplicial es un $0$-simplejo; El conjunto de  $n$-simplejos de $K$ será denotado por $K_{n}$. 
	
	La dimensión de $K$ es $\dim(K)=\max\{\dim(\sigma)\mid\sigma\in K\}$. Una \emph{faceta} (o \emph{careta} si se usa cara) de un complejo simplicial es un simplejo máximal con respecto a la contención; es decir, $\sigma$ es faceta si $\sigma\subseteq\tau$, entonces $\tau=\sigma$. Diremos que un complejo simplicial es \emph{puro} wsiempre que todas sus facetas tengan la misma dimensión.
	
	Los complejos simpliciales son objectos combinatorios. Usualmente se representan gráficamente como sigue. Un punto representa un vértice, las aristas,  $1$-simplejos; los triángulos representan $2$-simplejos, etc. En la Figura~\ref{fig:eg_simplex} encontramos ejemplos de simplejos y en la Figura~\ref{fig:eg_simplicial_complex}, de complejos simpliciales. Nótese que el triángulo hueco en la Figura~\ref{fig:eg_non_pure} no es un simplejo. También, es importante notar que del dibujo solo es imposible determinar si el tetraedro de la Figura~\ref{fig:3_simplex} es un $3$-simplejo o es la unión de 4 $2$-simplejos. Es por esto último que sólo representamos con fines ilustrativos.
	
	\begin{figure}[h]
		\centering
		\begin{subfigure}{.3\textwidth}
			\centering
			\subimport{images/}{eg_1_simplex}
			\caption{Un $1$-simplejo.}
			\label{fig:1_simplex}
		\end{subfigure}
		\begin{subfigure}{.3\textwidth}
			\centering
			\subimport{images/}{eg_2_simplex}
			\caption{Un $2$-simplejo.}
			\label{fig:2_simplex}
		\end{subfigure}
		\begin{subfigure}{.3\textwidth}
			\centering
			\subimport{images/}{eg_3_simplex}
			\caption{Un $3$-simplejo.}
			\label{fig:3_simplex}
		\end{subfigure}
		\caption{Representación gráfica de alguno simplejos. el tetraedro en \subref{fig:3_simplex} es un sólido $3$-dimensional.}
		\label{fig:eg_simplex}
	\end{figure}
	\begin{figure}[h]
		\centering
		\begin{subfigure}{.4\textwidth}
			\centering
			\subimport{images/}{eg_octahedral_sphere}
			\caption{A pure simplicial complex.}
			\label{fig:eg_octahedral_sphere}
		\end{subfigure}
		\begin{subfigure}{.4\textwidth}
			\centering
			\subimport{images/}{eg_non_pure_simplicial_complex}
			\caption{A non-pure simplicial complex.}
			\label{fig:eg_non_pure}
		\end{subfigure}
		\caption{Examples of simplicial complexes. All triangles in \subref{fig:eg_octahedral_sphere} are filled but the simplicial complex itself is not a $3$-dimensional solid.}
		\label{fig:eg_simplicial_complex}
	\end{figure}
	
	Sea $\sigma$ un simplejo. Vamos a sobrecargar la notación y usaremos $\sigma$ para denotar al complejo simplicial $\mathcal{P}(\sigma)$ como a la única faceta de tal complejo simplicial. En este sentido, todo simplejo es un complejo simplicial.
	
	\begin{exercise}
		¿Cuántos $k$-simplejos tiene un $n$-simplejo?
	\end{exercise}
	
	El \emph{complejo simplicial generado} por $F\subseteq\mathcal{P}(V)$ es el complejo simplicial mínimo con respecto a contener todos los simplejos $\sigma\in F$. Tal complejo simplicial es denotado por 
	\[
	\bigcup F=\bigcup_{\sigma\in F}\sigma.
	\]
	
	\begin{exercise}
		Muestra que todo complejo simplicial es generado por sus facetas.
	\end{exercise}
	
	Un \emph{subcomplejo} de un complejo simplicial $K$ es un subconjunto de $K$ que es un complejo simplicial. El \emph{$n$-esqueleto} de $K$ es el subcomplejo de $K$ generado por todos los $k$-simplejos en $K$ con $k\leq n$ y es denotado por $\operatorname{skel}_{n}(K)$.
	
	\begin{exercise}
		Muestra que $\operatorname{skel}_{n}(K)$ es puro de dimensión $n$.
	\end{exercise}
	
	\begin{exercise}
		Muestra que $\operatorname{skel}_{n}(K)=K_{n}$ si y sólo sí $n=-1$ o $K$ es vano. ¿Qué pasaría con esta equivalencia si en la Definición~\ref{def:simplicial_complex} eliminamos los complejos simpliciales vacío y vano?
	\end{exercise}	
	
	Un \emph{simplejo frontera} de un $k$-simplejo $\sigma$, es un $k-1$-simplejo $\tau\in\sigma$. La \emph{frontera} de $\sigma$ es $\partial\sigma=\bigcup\{\tau\mid\tau\text{ es un simplejo frontera de }\sigma\}$.
	
	\begin{exercise}
		$\partial\sigma$ es un complejo simplicial.
	\end{exercise}
	
	\section{Realización geométrica}
	\noindent Intuitivamente, el complejo simplicial de la Figura~\ref{fig:eg_octahedral_sphere} es una $2$-esfera. Esta sección formaliza tal intuición. Usamos $\mathbb{R}^{V}$ para denotar el $\mathbb{R}$ espacio vectorial con base $V$, en particular, $\mathbb{R}^{n}$ es el espacio euclidiano de $n$ dimensiones. Siempre consideramos $\mathbb{R}^{V}$ como espacio topológico con la topología inducida por la métrica euclidiana.
	
	\begin{definition}\label{def:affine_ind_set}
		Un \emph{conjunto afinmente independiente} en $\mathbb{R}^{n}$ es un conjunto $A=\{v_{0},\ldots,v_{k}\}$ tal que para cada $t_{i},s_{i}\geq0$ con 
		\[
		\sum_{i=0}^{k}s_{i}=\sum_{i=0}^{k}t_{i}=1,
		\]si 
		\[
		\sum_{i=0}^{k}s_{i}v_{i}=\sum_{i=0}^{k}t_{i}v_{i},
		\]entonces $s_{i}=t_{i}$ para cada $i$. Un vector $x\in\mathbb{R}^{n}$ es una \emph{combinación afín} de $A$ si $x=\sum_{i=0}^{k}s_{i}v_{i}$ con $\sum_{i=0}^{k}s_{i}=1$ y $s_{i}\geq0$ para cada $i$. La \emph{cápsula convexa} (\emph{convex hull} en inglés) de $A$ es el conjunto de todas las combinaciones afines de $A$.
	\end{definition}
	
	\begin{exercise}
		Demuestra que todo conjunto linealmente independiente de $\mathbb{R}^{n}$ es afín independiente.
	\end{exercise}
	
	Tomamos las definiciones~\ref{def:standard_V_simplex} y~\ref{def:geom_real} de \cite[Section 2.2.1]{kozlov:2008:combinatorial:alg:topo}.
	
	\begin{definition}\label{def:standard_V_simplex}
		Sea $V$ un conjunto finito. El \emph{$V$-simplejo estándar} es la cápsula convexa de la base ortnonormal estándar de $\mathbb{R}^{V}$. 
	\end{definition}
	En la Figura~\ref{fig:eg_standard_2_simplex}, tenemos el $V$-simplejo estándar para $V=\{x_{0},x_{1},x_{2}\}$. 
	\begin{figure}
		\centering
		\import{images/}{eg_standard_2_simplex}
		\caption{El $V$-simplejo estándar para $V=\{x_{0,}x_{1},x_{2}\}$.}
		\label{fig:eg_standard_2_simplex}
	\end{figure}
	
	\begin{definition}\label{def:geom_real}
		Sea $K$ un complejo simplicial. Consideremos la unión de todos los $\sigma$-simplejos estándar en $\mathbb{R}^{K_{0}}$ con $\sigma\in K$. Este conjunto con la topología de subespacio de $\mathbb{R}^{K_{0}}$ es la \emph{realización geométrica estándar} de $K$.  Cualquier espacio homeomorfo a la realización geométrica estándar de $K$ como $|K|$ y la llamaremos la \emph{realización geométrica} de $K$.
	\end{definition}
	
	\begin{remark}\label{rem:geom_realization}
		La realización geométrica es definida mediante los $V$-simplejos estándar. Como ellos tienen la topología de subespacio de $\mathbb{R}^{V}$ siempre supondremos que $|K|$ tiene una métrica que le induce la topología adecuada. También, cualquier afirmación topológica hecha sobre un complejo simplicial estará hecha sobre su realización geométrica.. 
	\end{remark}
	
	\begin{exercise}\label{rem:geom_realization_colimit}
		Sea $K$ un complejo simplicial. En este ejercicio veremos cómo podemos estudiar complejos simpliciales infinitos y sus realizaciones geométricas. La equivalencia de la última parte sirve como definición de la topología de la realización geométrica de complejos simpliciales infinitos \cite[p. 8]{munkres:1984:algebraic:topology}.
		\begin{enumerate}
			\item Prueba que si $\sigma\in K$, entonces $|\sigma|\subseteq |K|$.
			\item Muestra que la realización geométrica de cualquier simplejo es un espacio compacto.
			\item Demuestra que si $\sigma\in K$, entonces $|\sigma|$ es un conjunto cerrado de $|K|$.
			\item Demuestra que $C\subseteq|K|$ es cerrado si y solo si $C\cap|\sigma|$ es cerrado para cada $\sigma\in\Delta$.
		\end{enumerate}
	\end{exercise}
	
	\section{Funciones simpliciales}
	\noindent Si $f\colon X\rightarrow Y$ es una función, usamos $f(A)$ para denotar la imagen de $A\subseteq X$ bajo $f$.
	\begin{definition}\label{def:simplicial_map}
		Dados dos complejos simpliciales $K$ and $L$, una \emph{función simplicial} de $K$ a $L$ es una función $f\colon K_{0}\rightarrow L_{0}$ tal que si $\sigma\in K$ entonces $f(\sigma)\in L$.
	\end{definition}
	Observe that a simplicial map induces a map $f\colon\Delta\rightarrow \Gamma$. In this way, when we say that $f\colon\Delta\rightarrow \Gamma$ is a simplicial map, we mean that it is induced by a simplicial map. It is easy to verify that the composition of two simplicial maps is a simplicial map and that the identity of $S_{0}(\Delta)$ induces the identity simplicial map of $\Delta$.
	
	\begin{remark}\label{rem:coeherent_gluing}
		Not every vertex map induces a simplicial map. Let $\sigma$ be a $2$-simplex. Despite $S_{0}(\sigma)=S_{0}(\partial\sigma)$, the identity map of $S_{0}(\sigma)$ does not induce a simplicial map from $\sigma$ into $\partial(\sigma)$. 
	\end{remark}
	
	\begin{figure}
		\centering
		\begin{subfigure}{.4\textwidth}
			\centering
			\import{images/}{eg_2_simplex}
			\caption{$\sigma$}
		\end{subfigure}
		\begin{subfigure}{.4\textwidth}
			\centering
			\import{images/}{eg_hollow_triangle}
			\caption{$\partial\sigma$}
			\label{fig:eg_hollow_triangle}
		\end{subfigure}
		\caption{The image of $\sigma$ under the identity of $S_{0}(\sigma)$ is not a simplex in $\partial\sigma$.}
		\label{fig:not_simplicial_map}
	\end{figure}
\end{document}