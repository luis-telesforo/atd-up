\documentclass{standalone}
\begin{document}
	Este capíyulo no pretende ser una introducción a Homología simplicial, sino que buscamos proveer las herramientas teóricas necesarias para poder estudiar aplicaciones de homología persistente. Por ello, recomendamos complementar la lectura de este capítulo con un curso de topología algebraica. En particular recomendamos \cite{munkres:1984:algebraic:topology} como referencia básica para topología algebraica. Otras referencias que pueden ser útiles para acompañar este capítulo son \cite[Capítulos IV y VII]{edelsbrunner:2010:computational} y \cite{scoville:2019:discrete:morse}. El primer texto es excelente para estudiar persistencia, pues los autores trabajan ampliamente en el tema. El segundo texto tiene la misma meta que este capítulo. 
	
	\section{Complejos de cadenas sobre $\mathbb{Z}_{2}$}
	
	La homología simplicial consiste de una familia de funtores de la categoría de complejos simpliciales a la de grupos. Aunque no vamos a revisar ningún concepto de teoría de categorías, sí necesitamos conocer la construcción de estos funtores. 
	
\end{document} 