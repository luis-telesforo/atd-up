\documentclass{standalone}

\begin{document}
	En este capítulo presentaremos una versión del Teorema (Lema) del Nervio. Este resultado es multifacético en el sentido de que hay diferentes enunciados no equivalentes pero que esencialmente dicen que algunas propiedades topológicas de un espacio son compartidas por un complejo simplicial  que se obtiene de una cubierta del espacio original.
	
	La razón para no proveer una prueba de este teorema es que cualquiera de sus enunciados requiere material de topología algebraica que no vamos a revisar. A pesar de ello, para entender su enunciado necesitamos diferentes conceptos que ilustraremos antes del teorema del nervio.  La mayoría de las pruebas que se omiten y no se dejan como ejercicio se pueden hallar en \cite{dieck:2008:algebraic:topology,munkres:1984:algebraic:topology,rotman:1988:algebraic:topology}. 
	
	\section*{Homotopía}
	El material presentado aquí no debe considerarse una introducción a este tema. Lo único que presentamos son las definiciones y algunos ejercicios para poder entender intuitivamente el Lema del Nervio. Para una comprensión más amplia recomendamos \cite{dieck:2008:algebraic:topology} o \cite{rotman:1988:algebraic:topology}.
	
	\begin{definition}\label{defn:path}
		Sean $X$ un espacio topológico, $x,y\in X$ y $I=[0,1]\subseteq\mathbb{R}$. Una \emph{trayectoria} en $X$ de $x$ a $y$ es una función continua $u\colon I\rightarrow X$ tal que $u(0)=x$ y $u(1)=y$. En este caso decimos que $x$ es \emph{conectable por trayectorias} con $y$
	\end{definition}
	
	\begin{example}
		La realización geométrica de una gráfica $K$ es conectable por trayectorias. Dados cualesquiera dos puntos en la realización geométrica o bien son vértices, o están en el interior de una arista o uno es un vértice y el otro está en el interior de una arista.
		
		Si los puntos son vértices, sabemos que existe un camino entre ellos en la gráfica. Así basta con parametrizar la realización geométrica de este camino para obtener una trayectoria en $|K|$.
		
		Si uno de ellos, digamos $x$, está en el interior de una arista $e$, podemos replicar el argumento anterior con el otro punto y cualquiera de los vértices de $e$. Finalmente, parametrizar la curva entre el último vértice y $x$.
		
		El otro caso es similar.
	\end{example}
	
	
	\begin{exercise}\label{ex:connected_comp}
		Usa lo siguiente para demostrar que ser conectable por trayectorias es una relación de equivalencia.
		\begin{itemize}
			\item La función constante es una trayectoria.
			\item La trayectoria inversa $u^{-}$ de $u$ es la composición $u\circ \operatorname{inv}$ donde $\operatorname{inv}(t) = 1-t$ para cada $t\in[0,1]$.
			\item La función que manda $t$ a $2t$ puede usarse para reparametrizar cualquier trayectoria de manera que la traza de $u$ se recorra en la mitad del tiempo.
		\end{itemize}
		Las clases de equivalencia bajo esta relación se llaman las componentes conexas por trayectorias de $X$.
	\end{exercise}
	
	\begin{definition}\label{defn:0_connected}
		El conjunto de las componentes conexas por trayectorias de $X$ es denotado por $\pi_{0}(X)$. Decimos que $X$ es \emph{$0$-conexo} si $\#\pi_{0}(X)=1$.
	\end{definition}
	
	
	\begin{definition}\label{defn:homotopy}
		Sean $X$ y $Y$ dos espacios topológicos. Dos funciones continuas $f,g\colon X\rightarrow Y$ son \emph{homotópicas} ($f\simeq g$) si existe una \emph{homotopía} $H$ de $f$ a $g$, es decir, si existe una función continua $H\colon X\times I\rightarrow Y$ tal que las siguientes ecuaciones entre restricciones de $H$ se cumplen: $H|_{X\times\{0\}}=f$ y $H|_{X\times\{1\}}=g$. Usualmente escribimos $H_{t}=H|_{X\times\{t\}}$.
	\end{definition}
	
	\begin{example}
		Sean $f,g\colon I\rightarrow\mathbb{R}^{2}$ definidas como $f(s) = (s,s^{2})$ y $g(s)=(s,s)$. Definamos $H\colon I\times I\rightarrow\mathbb{R}^{2}$ como $H(t,s) = (s, ts^{2}+(1-t)s)$. $H$ es continua pues sus funciones coordenadas son continuas en ambas variables. Luego $H_{0} = g$ y $H_{1} = f$. Por lo tanto, $f\simeq g$. Intuitivamente, $H$ está deformando la gráfica de la identidad en la de la función $x\mapsto x^{2}$ en el intervalo $I$.
	\end{example}
	
	\begin{definition}
		Una homotopía como en el ejemplo anterior se llama \emph{homotopía lineal}.
	\end{definition}
	
	\begin{exercise}\label{rem:homotopy_rel_equiv}
		Muestra, como en el Ejercicio~\ref{ex:connected_comp}, que ser homotópicas es una relación de equivalencia en las funciones continuas entre dos espacios topológicos. Formalmente debes tener cuidado puesto que las homotopías son funciones de dos variables, por lo que la continuidad se debe verificar utilizando la topología del producto: Un argumento de cálculo multivariado es suficiente.
	\end{exercise}
	
	\begin{exercise}\label{prop:homotopies_preserve_composition}
		Si $f\simeq f'$ y $g\simeq g'$ con $f\colon X\rightarrow Y$ y $g:Y\rightarrow Z$, entonces $g\circ f\simeq g'\circ f'$. Como antes, un argumento de cálculo multivariado ayudará a mostrar que la función que propones es continua.
	\end{exercise}
		
	
	\begin{definition}\label{defn:homotopy_equivalence}
		Un \emph{inverso homotópico} de una función continua $f:X\rightarrow Y$ es una función continua $g:Y\rightarrow X$ tal que $f\circ g$ y $g\circ f$ son homotópicas a la identidad. En tal caso, diremos que $f$ es una \emph{equivalencia homotópica} y $X$ y $Y$ serán llamados \emph{homotópicamente equivalentes} o \emph{del mismo tipo de homotopía}. Si $X$ es homotópicamente equivalente a un punto, diremos que es \emph{contráctil}.
	\end{definition}
	
	\begin{exercise}
		Muestra que todo subconjunto convexo de $\mathbb{R}^{n}$ es contráctil. Pista: puedes suponer que el origen está contenido en el subconjunto.
	\end{exercise}
	
	\begin{exercise}
		Muestra que $X$ es contráctil si y solo sí, $1_{X}$ es homotópica a una constante.
	\end{exercise}
	
	\begin{exercise}
		Un \emph{anillo} es el conjunto de puntos que se encuentran entre dos círculos concéntricos; en otras palabras un conjunto de la forma
		\[
		A = \{(x,y) \mid r\leq \|(x,y)\|\leq R\}
		\] Con este ejercicio demostrarás que un anillo es homotópicamente equivalente a $S^{1}$. Para ello supondremos que tenemos un anillo como arriba en el que $r=1$.
		
		\begin{enumerate}
			\item Considera la función $H\colon A\times I\rightarrow S^{1}$ definida mediante $H(x,y, t) = \frac{t(x,y)}{\|(x,y)\|} + (1-t)(x,y)$. Demuestra que esta función es una homotopía entre $1_{A}$ y $H_{1}$. 
			\item Si $p = H_{1}$ y $i\colon S^{1}\rightarrow A$ es la inclusión, demuestra que $p\circ i = 1_{S^{1}}$.
			\item Muestra que existe una homotopía $\bar{H}$ entre $i\circ p$ y $1_{S^{1}}$. Concluye que $S^{1}$ es homotópicamente equivalente a $A$ y por tanto, $A$ no es contráctil.
		\end{enumerate}
	\end{exercise}
	
\end{document}